\section{Speed}

\begin{quote}
{\small Placeholder text. }
\end{quote}

In this essay we study Branching Random Walks (BRWs) with selection (also called $N$-branching random walks), which we can think of as a cloud of particles evolving on $\R$ indexed by discrete time according to two mechanisms: 
\begin{enumerate}[1]
\item \vspace{-2mm}\textbf{branching} Each particle gives birth to its offspring around itself, according to some point process. \\
\item \vspace{-6mm}\textbf{selection} Out of all children of the current generation, the rightmost $N$ are selected to form the next generation. \\
\end{enumerate}
\vspace{-5mm}It will be convenient to think of BRWs and $N$-BRWs as stochastic processes taking values in the set $\frak{M}$ of counting measures $\mu$ on $\R$ which put non-negative integer mass on every real number and further satisfy $\mu([x, \infty)) < \infty$ for all $x \in \R$. The latter condition is needed for the phrase `rightmost particles' to be meaningful. We will write $\frak{M}_N \subset \frak{M}$ for measures which have total mass $N$ and $\delta_{x_0} \in \frak{M}_1$ for the unit mass at $x_0$. The interpretation is that if $\mu$ is the value of the ($N$-)BRW at some time $n$, then there are exactly $\mu(\{x\})$ particles at position $x$ at time $n$. There is a natural partial order on $\frak{M}$: we say that $\mu \preceq \nu$ if $\mu([x, \infty)) \leq \nu([x, \infty))$ for all $x \in \R$. For random elements $\scr{L}, \scr{G}$ of $\frak{M}$ (such as BRWs) we say that $\scr{L} \preceq \scr{G}$ if there exists a coupling $(\scr{L}, \scr{G})$ such that $\scr{L} \preceq \scr{G}$ almost surely. In summations involving $\scr{L}$ we'll write $\sum_{l \in \scr{L}} [\cdots]$ for the sum over positions $l$ of the particles in $\scr{L}$. We will write $\#\scr{L}$ for the total mass of $\scr{L}$. We'll further write $\scr{L}(k)$ for the random variable distributed as follows: if $\scr{L}$ has at least $k$ particles then set $\scr{L}(k)$ to be the position of the $k$-th particle from the right; otherwise set it equal to $\min \scr{L}$. \\

We now construct ($N$-)BRWs in great generality. Suppose that $\scr{L}$ is a random element of $\frak{M}$ and that $X \defeq (X_n)_{n\geq0}$ is an ($N$-)branching random walk evolving according to the law of $\scr{L}$. Suppose that $X$ is given up to time $n\geq0$. Define 
\begin{equation}\label{eqn:GeneralConstruction}
\widetilde{X}_{n+1} \defeq \sum\limits^{\# X_n}_{j=1} \sum\limits_{l \in \scr{L}_{n, j}} \delta_{X_n(j) +  l}. 
\end{equation}
If $X$ is a regular BRW just set $X_{n+1} \defeq \widetilde{X}_{n+1}$, in the $N$-BRW case set $X_{n+1}$ to be the $N$ rightmost particles of $\widetilde{X}_{n+1}$. This construction allows for a natural and important coupling between ($N$-)BRWs. This coupling was first described in \cite{exptails}, the way we present it here is more general and similar to \cite{mallein2018n} Lemma 4.1.

\begin{lemma}\label{lem:monotonicity}
Let $1 \leq N_1 \leq N_2$ and $\mu_i \in \frak{M}_{N_i}$ for $i = 1,2$. Consider random elements $\scr{L}_i \in \frak{M}_{N_i}$ for $i = 1,2$. Then if $(X^{(i)}_n)_{n\geq0}$ is a(n) ($N$-)BRW which evolves according to the law of $\scr{L}_i$ and starts from $\mu_i$ respectively, then there exists a coupling such that $X^{(1)}_n \preceq X^{(2)}_n$ almost surely for all $n \geq 0$. 
\end{lemma}

\begin{proof}[Sketch of proof]
We construct the coupling inductively. The base case clearly holds. Given $X^{(1)}_n \preceq X^{(2)}_n$, independently take $N_2$ i.i.d. copies $\{(\scr{L}^{(1)}_i, \scr{L}^{(2)}_i)\}^{N_2}_{i=1}$ of the coupling of $\scr{L}_1$ and $\scr{L}_2$ that witnesses the partial order. Using these, construct $\widetilde{X}^{(1)}_{n+1}$ and $\widetilde{X}^{(2)}_{n+1}$ as in (\ref{eqn:GeneralConstruction}). If the $X^{(i)}$ are regular BRWs just set $X^{(i)}_{n+1} = \widetilde{X}^{(i)}_{n+1}$, if they are $N$-BRWs take the rightmost $N$-particles like before. Either way, we have $X^{(1)}_{n+1} \preceq X^{(2)}_{n+1}$ as desired. 
\end{proof}	

% \begin{remark}
% Notice that we could easily modify the discussion above to allow for the point process $\scr{L}$ to depend on the generation of the ($N$-)BRW, i.e. at generation $n \geq 0$ the children of the particles would follow a law $\scr{L}_n$ dependent on $n$. 
% \end{remark}

