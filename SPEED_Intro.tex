\section{Speed}

\begin{quote}
{\small Placeholder text. }
\end{quote}

In this essay we study Branching Random Walks (BRWs) with selection (also called $N$-branching random walks), which we can think of as a dynamic cloud of particles on $\R$ indexed by discrete time. Branching random walks with selection evolve according to two mechanisms 
\begin{enumerate}[1]
\item \vspace{-2mm}\textbf{branching} Each particle gives birth to its offspring around itself, according to some point process. \\
\item \vspace{-6mm}\textbf{selection} Out of all children of the current generation, the rightmost $N$ are selected to form the next generation. \\
\end{enumerate}
\vspace{-5mm}It will be convenient to think of BRWs and $N$-BRWs as stochastic processes taking values in the set $\frak{M}$ of counting measures $\mu$ on $\R$ which put non-negative integer mass on every atom and further satisfy $\mu([x, \infty)) < \infty$ for all $x \in \R$. The latter condition is needed for the phrase `rightmost particles' to be meaningful. We will write $\frak{M}_N \subset \frak{M}$ for measures which have total mass $N$ and $\delta_{x_0} \in \frak{M}_1$ for the unit mass at $x_0$. The interpretation is that if $\mu$ is the value of the ($N$-)BRW at some time $n$, then there are exactly $\mu(\{x\})$ particles at position $x$ at time $n$. There is a natural partial order on $\frak{M}$: we say that $\mu \preceq \nu$ if $\mu([x, \infty)) \leq \nu([x, \infty))$ for all $x \in \R$. Naturally, for random elements $\scr{L}, \scr{G}$ of $\frak{M}$ (such as BRWs) we say that $\scr{L} \preceq \scr{G}$ if there exists a coupling $(\scr{L}, \scr{G})$ such that $\scr{L} \preceq \scr{G}$ almost surely. For $\scr{L} \in \frak{M}$ we also define $\{ l \mid\,l \in \scr{L}\} \defeq \{ I \subset \R \mid\, \sum_{i \in I} \delta_{i} = \scr{L}\}$. \\

Using the notation introduced above, we can construct $N$-branching random walks in great generality. Suppose that $\scr{L}$ is a random element of $\frak{M}$ and that $X \defeq (X_n)_{n\geq0}$ is an $N$-branching random walk evolving according to the law of $\scr{L}$. Then $X$ is inductively constructed as follows: given $X_n \in \frak{M}_N$ for $n \geq 0$, take $N$ i.i.d. copies $(\scr{L}_i)^N_{i=1}$ of $\scr{L}$ independently of $X_n$. Writing $X_k(1) \leq \cdots \leq X_k(N)$ for the particles of $X_k$ for all $k \in \N$, we let 
\begin{equation}\label{eqn:GeneralConstruction}
\tilde{X}_{n+1} = \sum\limits^N_{i=1} \sum\limits_{l \in \scr{L}_i} \delta_{X_n(i) +  l}, 
\end{equation}
and define $X_{n+1}$ to be the rightmost $N$ particles in $\tilde{X}_{n+1}$. This construction allows for a natural and important coupling between ($N$-)BRWs. This coupling was first described in \cite{exptails}, the way we present it here is more general and similar to \cite{mallein2018n} Lemma 4.1.

\begin{lemma}\label{lem:monotonicity}
Let $1 \leq N_1 \leq N_2$ and $\mu_i \in \frak{M}_{N_i}$ for $i = 1,2$. Consider random elements $\scr{L}_i \in \frak{M}_{N_i}$ for $i = 1,2$. Then if $(X^{(i)}_n)_{n\geq0}$ is a(n) ($N$-)BRW which evolves according to the law of $\scr{L_i}$ and starts from $\mu_i$ respectively, then there exists a coupling such that $X^{(1)}_n \preceq X^{(2)}_n$ almost surely for all $n \geq 0$. 
\end{lemma}

\begin{proof}[Sketch of proof]
We construct the coupling inductively. Given $X^{(1)}_n \preceq X^{(2)}_n$, independently take $N_2$ i.i.d. copies $\{(\scr{L}^{(1)}_i, \scr{L}^{(2)}_i)\}^{N_2}_{i=1}$ of the coupling of $\scr{L}_1$ and $\scr{L}_2$ that witnesses the partial order. Using these, construct $\tilde{X}^{(1)}_{n+1}$ and $\tilde{X}^{(2)}_{n+1}$ as in (\ref{eqn:GeneralConstruction}). If the $X^{(i)}$ are regular BRWs just set $X^{(i)}_{n+1} = \tilde{X}^{(i)}_{n+1}$, if they are $N$-BRWs take the rightmost $N$-particles like before. Either way, we have $X^{(1)}_{n+1} \preceq X^{(2)}_{n+1}$ as desired. 
\end{proof}