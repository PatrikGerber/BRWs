\section{Effect of selection on ancestry: an exactly soluble case and its phenomenological generalization}\label{dec:}

\begin{quote}
{\small Placeholder text. }
\end{quote}

\section{The space $D$}
This section is based on Sections 3,11 and 12 of \cite{jacod2013limit}. Consider $D(I, \R)$ where $I$ is a closed interval. 
Let $\Lambda$ be the space of continuous bijections from $[0,1]$ to itself which are zero at zero. The $J_1$ topology is defined by the metric
\begin{equation}
d_{J_1}(f,g) = \inf_{\lambda \in \Lambda} \{ \norm{f \circ \lambda}_{\infty} \lor \norm{\lambda - Id}_{\infty} \}, 
\end{equation}
where $Id : x \mapsto x$. The $M_1$ topology is defined by the same metric but on the completed graph. Note that The topology $J_1$ is stronger. These definitions extend to $I \ [0, \infty)$ by saying that $x_n \in D$ converges to $x \in D$ if convergence happens in $D([0, t])$ for all continuity points $t$ of $x$. The topology this defines is metrisable (see page 83 of \cite{jacod2013limit}). The topologies they induce are called strong and sometimes written $SJ_1$ and $SM_1$. We talk about weak topologies when we consider $D(I, \R^k)$ with $k > 1$ and they are equal to the product topology, but they are not important to our discussion. \\

The $(S)J_2$ and $(S)M_2$ topologies are the ones induced by applying the Hausdorff metric to the functions graphs and completed graphs respectively. 


\section{Background}
\begin{theorem}[{{\cite[Theorem 7.4.1.]{durrett2010probability}}}]
Let $\{X_{m,n} \mid\, 0 \leq m < n\}$ be a family of random variables satisfying 
\begin{enumerate}[(i)]
\item $X_{l,n} \leq X_{l,m} + X_{m, n}$ for all $l < m < n$
\item The distribution of $\{X_{m + k, n + k} \mid\, 0 \leq m < n\}$ does not depend on $k \in \N$
\item $\Ex{X_{0,1}^+} < \infty$ and there exists $\gamma > -\infty$ such that $\Ex{X_{0,n}} > \gamma n$ for all $n \in \N$. 
\end{enumerate}
Then 
\end{theorem}

\section{exactly soluble}
The model we consider is constructed as follows: The individual at position $x_0 \in \R$ has offspring according to a poisson process with intensity measure that has density $x \mapsto \exp(- (x - x_0))$ with respect to the Lebesgue measure. Out of the offspring of all individuals, the $N$ right-most are selected to form the population of the next generation. By additivity of Poisson processes, the location of the offspring is described by a Poisson process with density $\Psi(x) \sum\limits_{i = 1}^N \exp(-(x - x_i))$ where $x_i$ denotes the position of the $i$'th individual in the current generation. 

% Denote by $X_{N+1}$ the $N+1$-st rightmost particle's poisition. We compute the density of $X_{N+1}$ using the properties of Poisson processes:
% \begin{align*}
% f_{X_{N+1}}(x) &= \lim\limits_{\Delta \to 0} \frac{\Pr{X_{N+1} \in [x, x + \Delta]}} {\Delta} \\
% 			   &= \lim\limits_{\Delta \to 0} \frac{1}{N!\, \Delta}(1 - \exp(-\int\limits_x^{x+\Delta} \Psi(u)\, du))\exp(-\int\limits_{x+\Delta}^\infty \Psi(u)\,du) (\int\limits_{x+\Delta}^\infty \Psi(u)\,du) ^ N \\
% 			   &= \frac{1}{N!} \Psi(x) (\int\limits_x^\infty \Psi(u)\,du) ^ N \exp(-\int\limits_x^\infty \Psi(u)\,du) 
% \end{align*}

Let us find the joint density of the $N+1$ rightmost particles. Let $x_1, ..., x_{N+1}$ denote them, with $X_{N+1} < X_i$ for $i \in [N]$ but with no particular order on $X_1, ..., X_N$. Then weby the mean value theorem for integrals we have: \\
\begin{align*}
\Pr{\bigcap\limits_{i = 1}^{N+1} \{ X_i \in [x_i, x_i + \Delta_i] \}} = \frac{1}{N!} \exp\left(-\int\limits_{x_{N+1}}^\infty \Psi(u)\,du\right) \prod\limits_{i = 1}^{N+1}\left\{ \Delta_i \Psi(x_i) \right\} (1 + o(1)). \\
\end{align*}
Thus, dividing by $\Delta_1 \times ... \times \Delta_{N+1}$ and taking the $\Delta_i$ go to zero, we get the density
\begin{equation*}
f_{X_1, ..., X_{N+1}}(x_1, ..., x_{N+1}) = \frac{1}{N!} \exp\left(-\int\limits_{x_{N+1}}^\infty \Psi(u)\,du\right) \prod\limits_{i = 1}^{N+1} \Psi(x_i).
\end{equation*}
Now we can marginalise to obtain the density of $X_{N+1}$:
\begin{equation*}
f_{X_{N+1}}(x) = \frac{1}{N!} \Psi(x) \left(\int\limits_x^\infty \Psi(u)\,du\right)^N \exp\left(-\int\limits_x^\infty \Psi(u)\,du\right). 
\end{equation*}
From here it is easy to see that conditional on $X_{N+1}$, the $X_i$ are independent with density
\begin{equation*}
f(x) = \frac{\Psi(x)}{\int\limits_{X_{N+1}}^\infty \Psi(u)\, du} \Ind_{\{X_{N+1} < x\}}. 
\end{equation*}

