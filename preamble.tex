\usepackage[utf8]{inputenc}
\usepackage[english]{babel}
\usepackage[a4paper,margin=1.2in,footskip=0.25in]{geometry}
\usepackage[dvipsnames]{xcolor}
\usepackage{amsthm}
\usepackage{amsmath}
\usepackage{amssymb}
\usepackage{mathtools}
\usepackage{bbm}
\usepackage{enumerate}
\usepackage{mathrsfs}
\usepackage{float}
\usepackage{chngcntr}
\usepackage{sectsty}
\usepackage{yfonts}
\usepackage{url}
\usepackage{wrapfig}
\usepackage{xfrac}
\usepackage{stmaryrd}
\usepackage{hyperref} % Uncommment to make references clickable
\usepackage{tikz}
\usepackage{minted} % For including Python code
\usepackage{tabularx}
\usepackage{setspace}
% This sets the line spacing
\setstretch{1}
% This sets the paragraph indent
\setlength{\parindent}{0cm}

% \definecolor{nice_blue}{RGB}{0,51,102}
% \definecolor{nice_blue}{RGB}{73, 115, 171}
\definecolor{niceBlue}{RGB}{48, 99, 165}
% \definecolor{nice_blue}{RGB}{61, 89, 171}
% \definecolor{nice_blue}{RGB}{65, 105, 225}
\definecolor{oxfordBlue}{RGB}{0, 33, 71}

% Configuring style of references
\hypersetup{
    colorlinks = True,
    allcolors = blue
}

% Makes equation numbering based on section
\counterwithin{equation}{section}

% Changing the styling of section and subsection titles. Comment to get default styling
\sectionfont{\centering\sffamily\scshape\color{black}}
\subsectionfont{\sffamily\color{black}}
\subsubsectionfont{\sffamily\color{black}}

%%%%%%%%%%%%%%%%%%%%%%%%%%% Defining environments %%%%%%%%%%%%%%%%%%%%%%%%%%%%%%%%%%%%%%%%
\newtheoremstyle{slimDefinitionStyle} % name
    {\topsep}                    	  % Space above
    {\topsep}                    	  % Space below
    {}			                   	  % Body font
    {}                           	  % Indent amount
    {\mdseries\scshape}			  	  % Theorem head font
    { ---}                         	  % Punctuation after theorem head
    {.5em}                       	  % Space after theorem head
    {}  % Theorem head spec (can be left empty, meaning ‘normal’)
\newtheoremstyle{slimTheoremStyle} 	% name
    {\topsep}                    	% Space above
    {\topsep}                    	% Space below
    {\itshape}                   	% Body font
    {}                           	% Indent amount
    {\mdseries\scshape}			 	% Theorem head font
    { ---}                         	% Punctuation after theorem head
    {.5em}                       	% Space after theorem head
    {}  % Theorem head spec (can be left empty, meaning ‘normal’)
\theoremstyle{slimTheoremStyle} % Comment this to get boldface theorem headers

\newtheorem{theorem}{Theorem}[section]
\newtheorem{proposition}{Proposition}[section]
\newtheorem{corollary}{Corollary}[theorem]
\newtheorem{lemma}[theorem]{Lemma}

% \theoremstyle{definition}
\theoremstyle{slimDefinitionStyle}
\newtheorem{definition}{Definition}[section]

\theoremstyle{remark}
\newtheorem{remark}{Remark}[section]
\newtheorem{assumption}{Assumption}
\newtheorem*{construction}{Construction}


%%%%%%%%%%%%%%%%%%%%%%%%%% Defining custom commands %%%%%%%%%%%%%%%%%%%%%%%%%%%%%%%%%%%%%%%
% Common fonts
\renewcommand{\cal}[1]{\mathcal{#1}}
\newcommand{\bb}[1]{\mathbb{#1}}
\renewcommand{\frak}[1]{\textfrak{#1}}
\newcommand{\scr}[1]{\mathscr{#1}}


% Common sets
\newcommand{\R}{\mathbb{R}}
\newcommand{\N}{\mathbb{N}}
\newcommand{\Z}{\mathbb{Z}}
\newcommand{\Q}{\mathbb{Q}}

% Expectation
\newcommand{\E}{\mathbb{E}}
\newcommand{\Ex}[1]{\mathbb{E}\left[ #1 \right]}
\newcommand{\ExCond}[2]{\mathbb{E} \left[\left. #1 \right| #2 \right]}

% Probability
\renewcommand{\P}{\mathbb{P}}
\renewcommand{\Pr}[1]{\mathbb{P} \left( #1 \right)}
\newcommand{\PrCond}[2]{\mathbb{P} \left( \left. #1 \right| #2 \right)}
\newcommand{\Ind}{\mathbbm{1}}

% Common distributions 
\newcommand{\dNorm}[2]{\mathcal(N)\left( #1, #2 \right)}
\newcommand{\dExp}[1]{\text{Exp} \left( #1 \right)}
\newcommand{\dBer}[1]{\text{Ber} \left( #1 \right)}
\newcommand{\dPo}[1]{\text{Poisson} \left( #1 \right)}
\newcommand{\dBin}[2]{\text{Binomial} \left( #1, #2 \right)}


% Miscellaneous math stuff
\newcommand{\defeq}{\vcentcolon=}
\newcommand{\eqdef}{=\vcentcolon}
\DeclarePairedDelimiter\ceil{\lceil}{\rceil}
\DeclarePairedDelimiter\floor{\lfloor}{\rfloor}
\DeclarePairedDelimiter\bbracket{\llbracket}{\rrbracket}
\DeclarePairedDelimiterX{\norm}[1]{\lVert}{\rVert}{#1}

\newcommand*\circled[1]{\tikz[baseline=(char.base)]{
            \node[shape=circle,draw,inner sep=2pt] (char) {#1};}}


